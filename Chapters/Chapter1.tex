% Chapter Template

\chapter{Introduction}\label{chapter:introduction} % Main chapter title

Early warning system (EWS) plays an important role to reduce disaster risk. It can prevent loss of life and reduce the economic and material impacts of hazardous events including disasters. To make it more effective, early warning system needs to be involved actively more people around and communicate at risk from a variety of hazards, facilitate public education and awareness of risks, broadcast messages and warnings proficiently and ensure that there is a constant state of preparedness and that early action is enabled (Fakhruddin, 2009). To organize Early Warning System disaster management is assigned their responsibility. The disaster management is dispersed at various levels of the nation such as national, regional and local level. Their responsibility is to function the EWS and ensure communication between all levels. Operating EWS involves assuring preparation, act and reply in the event of a natural disaster to protect people in the time of danger (Mercy Corps & Practical Action, 2010). The use of EWS in natural disasters is often distributed from the national level while the local perspective is not properly studied. Our thesis will therefore consider contextual factors in relation to the locals’ perspective of EWS. 
Natural disasters are the process where EWS was to minimize the complications in unforeseen circumstances. The main concept of disaster management involves national and local government and engagement of the agencies to handle the EWS.  The main concept behind EWS is used to decrease the complications in the uncontrollable situations. The disaster management schemes are based upon the concept which  handles national and local government  and their engagement agencies  to manage the EWS. These candidates have appointed responsibilities within the EWS. To attain the general scenario, it is highly beneficial to present how EWS can be used in the rural areas.
 The context where EWS is used to decrease complications in uncontrollable situations. The concept of disaster management involves national and local government and engagement of agencies to manage the EWS. These participants have appointed responsibilities within the EWS. To achieve a general perspective, it is beneficial to present how EWS is used in rural areas.


%----------------------------------------------------------------------------------------
%	SECTION 1
%----------------------------------------------------------------------------------------

\section{Natural Disaster}\label{sec:natural_disaster}

% It is a good idea to have each sentence on a separate line, so that if you get feedback or changes from someone else
% the diffs will be much easier to manage
A natural hazard  is a natural process  which causes great destruction to those endangered.  Floods, fires, landslides and earthquakes are the most common natural hazards which cause major damage to the society (Alfieri, 2010). The most common natural hazards with the ability to cause major harm are floods, fires, landslides and earthquakes. The natural hazard becomes a natural disaster when serious damage affects a community. Serious damage includes loss or impact of individuals, belongings or environment highly affecting the community. Consequently, a disaster  is defined as the result of a natural hazard in amalgamation with vulnerable conditions and insufficient capability to lower down the negative consequences of the event (United Nations, 2009).
 The disaster management is assigned with various responsibilities with proper function to  help the communities at the time of the crisis. The disaster management  tries to implement an information system to predict and analyze the natural disasters in advance. In relation to flood, an information system is combined  with the  main rivers of a country to get benefit and provide information for the disaster management. This information system is helpful to predict flood events  and try to overcome risk in advance to handle the natural disaster. The advantage of this projected the local a time  to get prepared for the flood and save their livelihood  and other family members by shifting to another safe place (Mercy Corps & Practical Action, 2010). Advanced preparedness gives therefore the communities more time to prepare and reduces the probability of a natural disaster.

%----------------------------------------------------------------------------------------
%	SECTION 2
%----------------------------------------------------------------------------------------

\section{Disaster Management}\label{sec:disaster_management}

During disaster whole charge is under disaster management.  This team manages all the resources and responsibilities during  a crisis such as a flood (Khan, Vasilescu, & Khan, 2008). If these responsibilities are handled in a better way, than the lives and livelihoods can be saved. The work of the disaster management includes  acting, strategies which are important  to get the optimistic response in addition to contribute in preparation. Also, the governments involved must monitor the management of the government, non-government, voluntary and private agencies. This includes organizing the best response in the event of a crisis. 
The responsibilities of the disaster management can be divided into the three categories  (Mercy Corps & Practical Action, 2010). These are:
\begin{enumerate}
\item Preparedness
\item Response
\item Recovery
\end{enumerate}

Preparedness is the process which defines how to grow  an handle all kinds of the natural disaster.Thus the objective of preparedness is to toughen the structure, events and necessary awareness to act appropriately in  the occasion of natural disasters. This is possible only with the help of professional response and recovery organizations. To be prepared for a disaster includes analyzing the level of risk, integration with EWS and the process of always being prepared for a natural disaster. An example of preparedness can be to store extra food or have an emergency pack with clothes and other important   belongings easy accessible (Mercy Corps & Practical Action, 2010). 
The Response  is the process which requires  immediate actions to save lives when natural disasters occur. This also needs emergency services and public help in  reporting of the disaster immediately. By hand over the community to safer areas and give the people basic needs, the most essential requirements are achieved. The focus of the response is to identify immediate and short-term needs when disasters occur (Gaire et al., 2015; Mercy Corps & Practical Action, 2010).
In some cases, the response integrated with the recovery phase. The reason behind this integration The reason for this is to provide an occasional help, for example, temporary housing, medicines, food and water supplies for the responses. These actions are significant to begin the action for the recovery segment. It is the restoration action which takes place when natural disaster occurs. In this segment, the task is to reestablish what has been lost or damaged. Another the focus is to implement developments to decrease the risk for the next natural disaster.
Thus disaster management is crucial to prepare and recover from the crises. To get the best results these operations should be implemented properly.

\subsection{Necessity of Early Warning Systems}\label{sec:necessity_of_early_warning}

EWS considers both a concrete system to monitor and warn about natural disasters and the required procedures surrounding these events. 
There are many countries in the world those are suffering from the flood due to the monsoon season.  Due to rain, most of the rural areas are affected annually. In this situation, an ESW is helpful to reduce the consequences of the floods. It helps to improve the communication between districts and inform people how  to react and save their belongings and other materials in advance (Wafi et al., 2015).   Risk reduction is the crucial element and EWS are precise for its warnings (Fakhruddin et al., 2015). In this research, our main concern is to provide warning signals for the tsunami or other natural disaster to detect in advance and provide the remedies for same. The concern is to look closer into how EWS is used and what influences the use of EWS. Therefore, three cases have been used to introduce the subject and acquire relevant knowledge about how EWS has been implemented in developing countries. The tsunami destroys various crops and affect the region in a bad way.   Early warning system helps to detect their lives and crops with preparing for extreme flood. This study explores the possibility of a 10-day prediction which can give farmers enough time to prepare for a harvest (Fakhruddin et al., 2015).
Flood is one of the major problems in the country like Australia as its climate  gives them two monsoon seasons. Sometimes monsoon and earthquakes become the major reasons for the tsunami which effects  all over the country badly. Mainly this natural disaster affects islands and country which are near to the sea or ocean. Prior literature explains that flood control measures have been tested with some success, but it has not been very effective in reducing the issue of flood. During natural calamities, they have understood that the flood cannot be  controlled from occurring. However, with the support from a flood EWS the loss of lives and harm to property can be reduced. This system integrated provides real-time information about the river’s water and provide the information to the Australian government regarding tsunami or other disasters. The EWS is applied to include involvement of disaster planning and management agencies. Organization and the dissemination of information in the occasion of the flood is accomplished through an internet-based system where different agencies have access to real-time information and can exchange relevant and reliable data (Billa, Mansor, & Mahmud, 2004).
A List of some issues as follows (Kubo,2019):

\begin{enumerate}
\item The impact of the tsunami due to the Great East Japan Earthquake was larger than what a map forecasted. 
\item The local government who took charge of the disaster prevention fails to gather information related to the tsunami affected area. Lifeline damage, road etc.
\item The rescue staff involved in the rescuing effort was hit by the tsunami(Kubo,2019).
 \end{enumerate}

Furthermore, there are some more problems  which need to be discussed under tsunami effects. These are:

\begin{enumerate}
\item To provide early warning signals for the tsunami, cyclones eruption or other natural disaster in the region.
\item To provide reasonable and resilient 2-way mode of the communications to the remote and not reachable places to give helps in the emergency  plans, responses  and to the day-to-day economic growth as well.
\end{enumerate}

It is better to focus on the framework to understand plans in a better way to achieve community resilience (Samamdipour, 2019). These problems are currently typically solved through a combination of:
\begin{enumerate}
\item It is expensive to handle the cost of the tsunami warning siren towers  in the remote areas which exceeds \$20,000 per kilometer of coastline.
\item The Installation cost of HF radio that varies from cost \$10,000 - \$20,000 each time. This cost includes training cost and replacement of  electronic equipment such as batteries, antennae. 
\item The shipping of the damage assessment teams is also expensive and costs in thousands of dollars.
\item  The cost of satellite based services and other traditional satellite can reach upto \$10,000 per day per device during any emergency situation.
\end {enumerate}
\subsection{The cycle of Early Warning system}\label{sec:cycle_of_ews}

To perform sufficient EWS it  is mandatory to have proper guidelines of how to conduct and work the processes.Through the following a set of guidelines, EWS  helps in a decision-making process during  the crisis (Zia, 2015). Thus, United Nation has developed recommendations which involve various considerations of four key elements when using EWS. These are Risk Knowledge, Monitoring and Warning, Dissemination of Information and Response Capabilities (United Nations, 2009).  Further, these capabilities are divided into complete cycle. There are numbers of the ways to organize this scheme into the complete cycle.
 
                                          
Fig.1.1  Life cycle of Early warning system
\subsection{Knowledge of Risk}\label{sec:Knowledge of risk}
 It concerns the groups and  identify of their own weakness towards  natural disaster. A risk assessment of the community and its vulnerability and contact with  flood must be directed  to know more about the risks (Smith et al., 2017). This assessment comprises  to check the water level and evaluate the levels  when it needs to warn the locals. Further,this part also needs an assessment of the locals’ knowledge to determine how much they are concerned about the risk and their practice to suddenly occur natural disaster (Mercy Corps & Practical Action, 2010). 
In the people-centered EWS, the local authorities are estimated  to have the awareness of the risk. It also needs  locals training to manage the system and to know the levels of the sea and what will happen if this level arise due to some unforeseen circumstances and other essential things of the EWS. This gives the local authorities the ability to correctly inform all individuals in the event of a flood, tsunami etc. Lastly, this element requires establishment of safe locations and routes for evacuation process, and must consider all individuals regardless of condition (Smith et al., 2017). 

\subsection{Monitoring and Warning Service}\label{sec: Monitoring and warning service}
 
 Monitoring includes the investigation of the area so that the pragmatic steps must be chosen for the better future. Monitoring includes the investigation of the area so that the pragmatic steps must be chosen for the better future. If you are analyzing and choosing manual warning system for EWS, and analyze risk by analyzing the critical places (Smith et al., 2017). Moreover, a community reader is a local person who is responsible to monitor the water level or rainfall. The main roile of the automation of the EWS is to provide information system which is installed to predict and forecast any uncommon  activity and increase level of the water. Mainly these system are handled by government authorities.  By keep investigating the river and other situation which can  cause natural disaster helps to  generate automatically and on the regular basis. The entire process bhelps to create the warning signals at the early stage and increase lead time and also aware the communities as well (Mercy Corps & Practical Action, 2010).
\subsection{Information Dissemination }\label{information_dissemination}
It considers the actual process of distributing the warning information. This element is closely integrated with the element above, to be able to generate critical and timely information.  In the case tsunami,  when the level of the water reach near or above the warning or danger level, thus this element of the EWS is important.  The communication is distributed by the community readers through phone calls using predefined communication charts. This chart consists of contact details of the cocern person such as  disaster management committee, security forces and local media (Smith et al., 2017).  Here in this case, top-down approach also followed.  Therefore, this process whole system is divided into numbers of the system. The first responsibility occurs at the district-level.  There dutu includes to maintain electronic display boards with sirens  and automated sirens to show risk at different level of the water (Smith et al., 2017). The second concept is SMS warning information.which enable SMS warning system. These messages are sent to the district officers, DHM offices  and security forces (Gautam & Phaiju, 2013). Whenever these persons get the information, their duty is to take the strigent action acooridng to their knowledge.  These elements of EWS also focus on the information of the warning to provide the possible help. Finally, the pragmatic steps are taken when all the parties are agreed upon the particular solution.  The element of the research are:
\begin{enumerate}
\item To contribute to improve the understanding and appropriate action by the communities. 
\item Prior research shows the importance of the information to be simple, useful and understandable for the recipient of the information and act according to the situation (Fakhruddin et al., 2015; Jaiswal & van Westen, 2013).
\end{enumerate} 
However, as these studies are on the same context, the factors might be recognized in the Australian  context, but variations are also believed to appear. 
\subsection{Response Capacity}\label{response_capacity}

This elements deal with the disaster management’s activities. During this process, it is believed that disaster manegments plans are  used to sort out the responsibilities and resources in case of natural disaster (Mercy Corps & Practical Action, 2010). These well-defined plans should be based upon the local abilities and their experience to assist the disaster management at the local and national level.  Test these plans frequently through trails to educate the local people for the correct results for the natural crisis (Smith et al., 2017).  It also needs extra preparations  to be performed by the local people  for the appropriate response. However, there is a high variation in the effort dependent on the engagement of the local authorities, causing different implementation approaches. These approaches vary from radio and posters to be integrated into the school curriculum (Gautam & Phaiju, 2013).
Risk and its consequences have been increased a lot from last a few decades. Risk has been dangerous, threatened and exposed. To handle the risk, each component should be eliminated to lower down the risk deduction (Slovic, 2004).  Risk management has taken all the measurements  to eliminate, control and regulate. It indicates the process of recognizing and minimizing the event effects that have known or unknown fraught with uncertainity. This involves the exchange of the risks received against potential benefits as well as the  scientific judgment against some factors or beliefs. They define us, how people should behave in the emergency situation, it totally depends upon the understanding and assessment of the risk measures (Anderson-Berry, 2003).
This term is handled differently by different journalists and a general public in term of definition, assessment and suitable response. 
This term is regarded differently by specialists and the general public in terms of definition, assessment, and suitable response. There is also a difference in risk perception. The awareness of disasters risk has been measured to be proportional to the threats and the undesirable significances of disasters. Some sources, includes “beliefs, attitudes, judgments, and feelings of the public, as well as the way cultural and social wider ones as threats to things that are of value to us (Sjöberg, 1999).
The natural disaster such as earthquake and tsunami puts a great impact of the communities and lives of the people. Therefore, it is a need to develop a system which can help to predict the upcoming tsunami and alert the people so that they can move to safer places (Mohanty, 2019).  There are various  growing recognition in disaster science to develop a framework which prone to natural hazards and  shape the resilience of the communities (George, 2010). ETC-Lali Low-Cost Communications (ETC) and All-Hazard Early Warning System are the two solutions which are highly useful to warn the people about the natural disasters. ETC Lali develops to  handle various  ubiquitous and pressing problems,  mainly present at the Indo-Pacific region, and for small Pacific Island Nations (PINs) in particular.
\subsection{Multiple Hazard warning systems}\label{multiple harzard warning systems}
Early Warning Systems (EWS) are highly useful for life-saving tool for floods, droughts, storms, bushfires, and other hazards.  It is estimated that economic losses linked to extreme hydro-meteorological events have increased  by 50 times over the past five decades, but the loss of the lives decreased significantly by a factor of about 10 (Benbrook, 2008). Thus, it has saved millions of the lives during this tenure. This has been attributed to better monitoring and forecasting of hydro-meteorological hazards and more effective emergency preparedness ( Haigh, 2018).
Effective EWS need four components:
\begin{enumerate}
\item Detection, analyzing and forecasting the hazards;
\item Investigates of risks involved;
\item Broadcasting of timely warnings - which should transmit  government authority
\item  Emergency Plans Activation to prepare and respond.
\end{enumerate}
These four components require coordinated from various agencies from national to local levels of the working system.  Failure of one  component or coordination, lack may lead to the failure of the entire system. To provide a warning for the natural disaster is our national responsibility. Thus, roles and duties of various public and private sector investors for implementation of EWS should be clarified and revealed in the national to local regulatory frameworks, planning, budgetary, coordination, and operational mechanisms (Hemingway, 2018).

\subsection{Expert Advisory Group on Multi-Hazard Early Warning Systems (EAG-MHEWS)}
The EAG-MHEWS been established by the WMO DRR Services Division for the purpose of the best practices, development guidelines and Training modules on Multi-Hazard Early Warning Systems (MHEWS). The already done carried out the best practices, according to the documentation for the MHEWS WMO based upon the guidelines on Disaster Risk Reduction and Institutional Partnership in MHEWS are under preparation. 
 Early warning gives a great contribution to the sustainable development. According to 2030 Agenda for the sustainable development, it addresses  early warning and give important role for the goals of the development, such as food security, healthy lives and resilient cities and climate change adapation. It is believed that it is one of the major focus area to enhance adaptive capacity, strengthen resilience and reduce loss and damage linked up with the adverse effects of climate change (Hemingway, 2018).
The checklist which provides is the result of the first Multi-hazard Early Warning Conference handles by the International Network for Multi-hazard Early Warning Systems (IN-MHEWS) 2 from 22 to 23 May 2017 in Cancún, Mexico (Kubo,2019). Later, this original document is keep updating and the final checklist is provided by the  Third International Conference on Early Warning: From Concept to Action, held from 27 to 29 March 2006 in Bonn, Germany.  By following its guidelines, people can save themselves from major disasters ( Haigh, 2018).
To  handle and understand existing problems, we need to understand what makes the context of the PINs unique. The primary challenges  can be plotted as:
 
\begin{enumerate}
\item Low GDP to coast-line length ratio, 
\item A High logistical burden to GDP ratio, and 
\item The requirement of skilled international labor and expensive imported products with excellent services
\end{enumerate}

 
\subsection{Challenges: The main challenges which are facing in the EWS are as follow:}

 \begin{enumerate}
\item Low GDP to Coastline Length Ratio: It is observed that a comparatively long coastline of PINs concerning GDP is the first challenge of this experiment. This ratio is marked an an important point because it depicts the monetary burden of any nation who is providing services to their coastlines.  Total GDP in Purchasing Power Parity (PPP) terms is used instead of GDP per capita. The reason behind this concept is to access the whole nation’s ability along with their entire coastline. 
\item High Logistical Burden to GDP Ratio: Installation and maintenance of logistics is also a strong barrier. There are many small and scattered islands are present in the Pacific Island Nations. For instance, Vanuatu has approximately 1,300km distance long from end to end and contains more than 70 islands. The main issue that faces in this coastline is the access to logistics which further increases its cost. Let’s take an example of groups of Vanuatu islands which are reachable in a boat only by several days, particularly during cyclonic weather. This problem becomes worst for country Federated States of Micronesia whose coastline is 8.7x times lengthier than the total land area and where hundreds of kilometers can part the islands. 
\item The requirement of skilled international labor and expensive imported products with excellent services:  Till date, there are numbers of the solutions has been proposed purely based upon the international vendors' services. These services are highly complex, which are not practically possible to use domestic labor for scheduling, organizing, functioning and maintaining the communication system.  Concerning this system, hardware and services for other operations such as installation, operation and maintenance must be purchased from the international retail organizations. As an outcome, skills transfer is limited; the local economy is neither fully grown nor expanded. Moreover, its flexibility and purchasing capacity also underestimate. It is not the right and significant way for the long term development and to make good relationship with the receiving countries.
\end{enumerate}

\subsection{Question Answers and Motivation}\label{Questions_Answers_and_Motivation}
There are number of factors  which are considered before implementing Early Warning Hazard System. We 
have collected some data for which people want to know their solution (Basher, R. (2006).

\begin{enumerate}
\item How will improvement takes place for the overall operation framework of the early warning system (Birkmann, 2010)
\item Please recognize and define calculation and  feedback process to improve the early warning system. Who will check it for the city and national level?
\item Which of the area given below are prioritized for the improvement of your EWS? (Basher, R. (2006).
•	Governance and Institutional Preparations national to local levels
•	Convey emergency planning and warnings at the city level
•	Hazard monitoring, forecasting, and mandates for warning development at the city level
•           Prepareness of Emergency and  its response activities
•	Teamwork among agencies at the national level and cooperation with local authorities and governments
\item Which areas of your country could benefit your country’s early warning system in urban and rural settings? (Birkmann, 2010)
\end{enumerate}
\subsection{ Motivation}
The engineering of EWS needs a  data gathering and sound analysis of the user requirements which are key points  for the success of the design and development.  Moreover, the methodology of engineering  helps to activate the seamless application of the general RM-TWS specifications for the engineering phase. It helps to provide the following tasks and specifications
\begin{enumerate}
\item  Refinement of  design artifacts step by step. It is a better idea to distinguish between an analysis, abstract design, summary, concrete design and engineering step.
\item  Collaboration for the necessities and capabilities. For analysis and design, an agile methodology must be followed design and engineering steps. It is helpful to fix the small bugs and deliver a small and a quality increment in the timebox.
\item  Documentation for the incremental architecture. It is recommended to increment the resulting TWS architecture with respect to the established reference model (Leffingwell, 2011).
Early warning for the natural disasters compromise various crises such as  eruptions, landslides, tsunamis and geological disasters, etc. In the term of Information and communication technology  gives early signals so that software and hardware get integrated  for the data processing, decision making and information dissemination to detect and analysis of imminent hazards and its related warnings as well. 
\end{enumerate}

\subsection{Evolution of  Early Warning Systems over Time}
The rapid development of the ICT has changed significantly from last decade in the process of business, industry, government and science.  The main question arisen  for the ICT revolutionized data and information management in the process of acquisition, processing, analysis and visualization of data as well as the integration of the computer system into the infrastructure. Following Moore’s (Schaller, 1997) and Kryder’s Law (Walter, 2005), change in process are driven by the contionous miniaturisation of the integrated circuits and increasing performance of the processing units, chips  and graphic units as well as reasonable storage capacities.  Local and wide area networks are the other  technological key drivers important for the telecommunication which are helpful to better know the understanding of the project. Other technological key drivers are the availability and performance of telecommunication facilities enabling local and wide area networks. 
 \subsection{2.1 Early detection and real-time reporting of deep-ocean tsunamis}
 
Eddie N. Bernard et.al (2001) discussed about the  early detection and real time reporting of the tsunamis occurred in the deep-ocean. In this paper, author represented a  method for detector sitting and also considered numbers of tradeoff between early tsunami detection, adequate source zone coverage  and DART system as well. In this proposed network author  displayed a design that will provide an adequate coverage of tsunamis which are orginated in source region  which is dangerous for U.S. Coastal communities.  
Mao Chongyuan et.al (2006)  mentioned various application  for monitoring and forecasting data mining algorithm with the help of data mining techniques.  The main focus in this paper is to use suitable data mining algorithm such as  aritifical neural networks, PCA, cluster analysis to evaluate  data  which is useful  for the  earthquake tsunami mechanism  and create model for the earthquake results. The main usage of this data is to generate quick impact to estimate model of tsunami to coastal areas.

\subsection{2.2Technology development for the forecasting for Tsunami Measuring and Forecasting}
Christian Meinig et.al (2005) discussed various enhancement in the Real-time Tsunami Measuring, Monitoring and Forecasting.  The technology used for the Tsunami forecasting is under the development of NOAA and is based upon welltested method utilized in various fore cast system  such as mixing of real-time measurement and modelling technologies.  This measurement helps for the examining and measurement of the sea-level data in the deep ocean handled by the seven-station network of the Deep-ocean Assessment and Reporting of Tsunami system.
Christian Meinig et.al (2007) introduced the Real-Time Deep-Ocean Tsunami Measuring, Monitoring, and Reporting System: The NOAA DART II Description and Disclosure. This paper helps to understand the components of the system which helps to ready second-generation Deep-Ocean Assessment and Reporting of Tsunamis system, known as DART II. The main purpose of this paper is to mention and reveal the existing DART II system characteristics in enough details of the others to start construction of additional deep-ocean tsunami detection and assessment systems which further consists of a critical portion of a tsunami forecast, cautionary and extenuation system.
\subsection{2.3 A web based online Tsunami detection system}
Mongkonkorn Srivachai et.al (2009) explained a web-based online tsunami warning system for the coastine of the Thailand’s coastline. Hypothetical earthquake and tsunami sources are extracted from the past experience and mainly focus in the Andaman microplate. The GRNN is useful to provide help to the future forecasting of wave heights for the variable location, magnitude and in-depth of the earthquake. The collected data defined the tsunami database warning system which is very useful for the Thailand region to save it in the future.
Amitabha Samajpati et.al (2011) introduced the Next Generation Mobile services using the application based upon the location specific.  The main advantages of using mobile phones with Built-in GPS which support location based service services over the approach using CELL-ID or Cell of Orgin solutions as well. It also discussed  a general architectural idea with the help of some examples and its application which are mainly implmented LBS  with the help of GPS enabled mobile devices on the customer side.
\subsection{2.4 A wireless ad hoc and sensor network architecture for situation management in disaster response}
Stephen M. George et.al (2010) discussed DistressNet: a wireless Ad Hoc and sensor network architecture for state management in disaster reaction. The awareness of the situation is critical to the current response. Disaster responders need  delivery of the high quality data at the accurate time to make the exact decisions. To meet the requirements, an adhoc wireless architecture  handle the disaster response with distributed collaborative sensing, structure-aware routing using a multi-channel protocol and exact localization of the resources.
Rone Il´ dio da Silva et.al (2010)  discussed the rate of disaster occurred and the people suffering has bene increasing day by day. After disaster incidence, the first of responder go the affected areas  to resuce people and resolve their problems. The communication network infrastructure is destroyed almost. For these details, it is significant to create a building for sensing of the environmental data to detect threats.
Avinash Srinivasan et.al (2010) described a Novel Secure Localization in Wireless Sensor Networks. The main operation of WSN is to sense and describe the  events which can be expressively adapted and answered to only if the precise position of the event is recognized. A straightforward solution is to implement  each sensor with a GPS headset that can provide the exact information about the information.
\subsection{2.5 GPS WATER LEVEL MEASUREMENTS FOR INDONESIA’S TSUNAMI EARLY WARNING SYSTEM}
Sch¨one, et. al (2011)  water level measurements for the early warning detection system for Indonesia. These sensors are well-suited for the offshore tsunami detection in the Indonesia with the help of advanced GPS technology for the water level managements. Utilization of the GPS approach is the new method. The concept of the GITEWS  includes GPS technology and ocean bottom measurements. This is basically used for the installations of the nearfield where seismic noise may reduced the OBP data, thus GPS_derived sea level gives the additional information.
Anna Dzvonkovskaya et.al (2009)  discussed the implementation of the Tsunami Detection using HF Radar WERA. The analysis of the implenentated data is observed by the HF radar for the long range in case of the tsunami travelling towards the cost is also mentioned.  This detection technique of the tsunami algorithm is based upon the order-static CFAR detection algorithm applied in the entropy field of the surface current. 
\subsection{2.6 Early Warning Solutions}
A. Mohanty et.al (2019) explained the  possible paradigm shift in the reaserch of disaster from varioustypes of the dangers. The main cocern of this paper is to develop the framework for the community resilence. This framework is highly useful for the early warning solution such as Flash flood, debris flow and land-sliding n various regions. The main objective of this paper is to discover the framework of the existing early warning systems for landslides, mudslides and to handle flood in the area of the Badkshan Province. The natural disaster such as earthquake and tsunami puts a great impact of the communities and lives of the people. Therefore, it is a need to develop a system which can help to predict the upcoming tsunami and alert the people so that they can move to safer places.Therefore the framework is based upon the household questionnaire survey with the help of sampling method. The Climate-related Disaster Community Resilience Framework (CDCRF) revealed that there diminishing resilience culture among the communities in conflict prone area of Badakhshan Mountain. Thus, this paper has been introduced the new scheme for the the Early warning solution so that the adequate help must be provided to the people on time and provide them the suitable help.
Haigh, R. et.al (2018) discussed about the development of the coastal regions which are densely populated and under threats of the natural disaster. The main aspect of the disaster risk reduction is the good performance of the multi-hazard early warning system (MHEWS) which need high level of the support from international and multilateral co-operation. They are made according to the well-defined operational standards which are implemented using broad range of the activities. This paper described the result of the first stage of the MHEW across Asia. This study is carried out by the HEIs across Europe and Asia along with several socio-economic actors in the region.  IN the initial study they focused on the five countries  for the development and analysis of the framework covering number of the dimension. This framework is developed with the help of consultation and need of the project fovusing on the five countries. The analysis findings will underpin later capacity building activities across the region.




















